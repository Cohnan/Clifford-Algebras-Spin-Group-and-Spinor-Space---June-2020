\section{Spinors}

%%%%%%%%%%%%%%%%%%%%%%%%%%%%%%%%%%%%%%%%%%%%%%%%%%%%%%%%%%%%%%%%%%%%%%%%%%%%%%%
\subsection{Spin Groups}

\begin{frame}{From now on\dots} % % % % % % % % % % % % % % % % % % %

    Let $n = dim_\RR V$, and $m = \lfloor \frac{n}{2} \rceil$.

    From now on we assume that \emph{$g$ is positive definite}, so that $V \cong \RR^n$ as inner-product vector spaces, and so we may write $v_1 \cdot v_2 := g(v_1, v_2)$ for $v_i \in V$. %(even in the big book Elements of Noncommutative Geometry this is assumed). 
    
    Let $\set{e_1, \dots, e_n}$ be an \emph{oriented} orthonormal basis.
    
    $\gamma = (-1)^m e_1 \cdots e_n$, but it is independent of the chosen oriented orthonormal basis. % In \RR^n: that the determinant of the matrix of columns is 1, not -1.

    We will concentrate on the \textbf{complex} Clifford algebra $\CC l(V)$, with hermitian inner product $\bra \cdot , \cdot \ket$.% Even if g is not positive definite, it is ISOMORPHIC, but not equal to the positive-definite case.
    
\end{frame}

\begin{frame}{Unit Vectors as Perpendicular Reflections} % % % % % % % % % % % % % % % % % % %
    Let \defn{$\cclv^\times$} be the group of invertible elements.
        \begin{itemize}
            
        \item Notice that, if $v \in V$, by the defining property of the Clifford algebra, $v^2 = ||v||^2$, we have that $v^\inv = \frac{v}{||v||^2}$.
            
        \end{itemize}
    
    Call an element $u \in \cclv$ \defn{unitary} if $u^* u = 1$. They are such that $\bra a, b \ket = \tau(a^* b) = \tau(a^*(u^* u) b)  = \bra ua, ub \ket$.
        \begin{itemize}
            
        \item If $x \in V$ is of unit length,  $x^* = \conj{(x^t)} = x = x^
        \inv$.
        
        \item Unitary vectors $w$ in $V^\CC$ \Iff $w = \lambda x$, where $\lambda \in U(1) =: \defn{\TT}$, $x$ is a unit vector in $V$.
        
        \item If $u$ is unitary \then $u \in \cclv^\times$.
            
        \end{itemize}
    

\end{frame}

\begin{frame} % % % % % % % % % % % % % % % % % % %

    For $a \in \cclv^\times$, let \defn{$Ad_a$} be the map $b \mapsto a b a^\inv$ for $b \in \cclv$. 
        \begin{itemize}
            
        \item This is a group representation, i.e. $Ad_{aa'} = Ad_a \circ Ad_{a'} \in GL(\cclv)$ 
            
        \end{itemize}

    \begin{lemma}
    For $x, v \in V$ ($g$ of any signature), $-Ad_x(v) = v - 2 \frac{g(x, v)}{g(x, x)} = $ ``\textbf{Reflection} of $v$ with respect to to the plane orthogonal to $x$''. So, $Ad_x \in O(V, g)$ when applied to $V\subset \cclv$.
    \end{lemma}
    \begin{proof} $-xvx^\inv = -(2g(x, v)x^\inv - v x x^\inv) = v - 2 \frac{g(x, v)}{g(x, x)}$
    \end{proof}
    
    Finally, it is not hard to see that \textbf{applying $2$ / an even amount of reflections is a rotation: $Ad_{x_1 x_2}\in SO(V, g)$}; \\the converse is also true: every rotation of $V$ may be written as an even product of reflections.
    
\end{frame}

\begin{frame}{$Spin(V)$} % % % % % % % % % % % % % % % % % % %

    \begin{Definition} $\defn{\spinv} := \set{x_1 \cdots x_{2k}: x_i \in V, ||x_i||^2 = x_i^* x_i = 1} \subset \clvg \subset \cclv$
    \end{Definition}
    
    We saw that $\spinv$ acts on $V$, via the adjoint representation, by rotations. Furthermore:
    
    \begin{proposition}
    The map $Ad|_{\spinv}: \spinv \to SO(V, g)$ is surjective.
    
    For any element $u \in \spinv$, both and $u, -u$ map to the same rotation. (i.e. $\spinv$ is the double cover of $SO(V, g)$)
    \end{proposition}
    
    $\spinv$ may be characterized as: $u \in \cclv$ is in $\spinv$ \iff $u$ is even, unitary and real ($u = \conj u$).
\end{frame}

\begin{frame}{Example: $\spin(2)$} % % % % % % % % % % % % % % % % % % %
    
    To see: \[\spin(2) = \set{x + y e_1 e_2 \,|\, x, y \in \RR;  x^2 + y^2 = 1} \cong \TT\]
    
    $\spin(2) \subset \set{x + y e_1 e_2 \,|\, x, y \in \RR;  x^2 + y^2 = 1}$:
        \begin{itemize}
            
        \item Product of $2$ unit vectors $\subset \set{x + y e_1 e_2}$: Let $u_1 = a e_1 + b e_2$ and $u_2 = \alpha e_1 + \beta e_2$ be normalized vectors in $\RR^2$. Then $u_1 u_2 = x + e_1 e_2 y \in \spin(2)$, with $x = a \alpha + b \beta$ and $y = a \beta  - b \alpha$ and $x^2 + y^2 = (a^2 + b^2)(\alpha^2 + \beta^2) = 1$.
        
        \item Even product of unit vectors $\subset$ Product of elements of $\set {x + y e_1 e_2} \subset \set {x + y e_1 e_2}$: $(x_1 + y_1 e_1 e_2)(x_2 + y_1 e_1 e_2) = x + y e_1 e_2$ with $x = x_1 y_1 - x_2 y_2$ and $y = x_1 y_2 + x_2 y_1$ and so $x^2 + y^2 = 1$.
            
        \end{itemize}
    
\end{frame}

\begin{frame}{Example (Cont.): $\spin(2)$} % % % % % % % % % % % % % % % % % % %
    
    (Cont.) To see: \[\spin(2) = \set{x + y e_1 e_2 \,|\, x, y \in \RR;  x^2 + y^2 = 1} \cong \TT\]
    
    $\spin(2) \subset \set{x + y e_1 e_2 \,|\, x, y \in \RR;  x^2 + y^2 = 1}$:
        
        \begin{itemize}
        
        \item Any element $x + y e_1 e_2$ can be seen as the product of unit vectors $e_1$ and $x e_1 + y e_2$: $e_1 (x e_1 + y e_2) = x e_1^2 + y e_1 e_2 = x + y e_1 e_2 \in \spin(2)$
            
        \end{itemize}
    \qed
    
    Looking at $\spin(2)$ as a ``subset'' of $M_2(\RR) \cong Cl(2)$, then $\spin(2) \cong \set{\begin{pmatrix} x & -y \\ y & x \end{pmatrix} \,|\, x^2 + y^2 = 1} \subset M_2(\RR)$.
    
\end{frame}

\begin{frame}{Example: $\spin(3)$ using Lie algebra} % % % % % % % % % % % % % % % % % % %

    The Lie algebra of $Spin(3) \subset Cl(\RR^3)$ is $Q(\bigwedge^\bullet \RR^3) = Cl^0(\RR^3) = \HH$, where $\hat i = e_2 e_3, \hat j = e_3 e_1, \hat k = e_1 e_2$.
    
    $Cl(\RR^3) \cong M_2(\CC)$ via the identification $e_j \mapsto \sigma_j$; so $\hat i \mapsto i \sigma_1, \hat j \mapsto i \sigma_2, \hat i \mapsto k \sigma_3$.
    
    The Lie algebra of $SU(2)$ is precisely that generated by $i\sigma_1, i\sigma_2, i\sigma_3$: they satisfy the commutation relations $[i\sigma_i, i \sigma_j] = 2 \epsilon_{ijk} (i \sigma_k)$; every element of $SU(2)$ may be written as $exp\{\frac{1}{2} \phi \, \vec n \cdot \vec{i \sigma}\}$.
    
    So, under the isomorphism $\CC l(\RR^3) \cong M_2(\CC) \oplus M_2(\CC)$ the Lie algebra of $\spin(3)$ is identified with the Lie algebra of \rtext{$SU(2)$ \Then $Spin(3) \cong SU(2)$}.
    
\end{frame}

\begin{frame}{$\spinv$ and $\spin^\uparrow(V)$ for general signature} % % % % % % % % % % % % % % % % % % %
    In Minkowsky spacetime $\spin(\bb M)$ has to be defined in a similar, but different way.
    
    \begin{definition}
    $\spin(V, g)$. Under the adjoint map, this is the double cover of $SO(V, g)$.
    \end{definition}
    
    \begin{definition}
    $\spin^\uparrow(V, g)$. This is the double cover of $SO^\uparrow(V, g)$.
    \end{definition}
    
    In the example of Minkowski spacetime, \textbf{$\spin^\uparrow(1, 3) \cong SL(2, \CC)$ is the double of $SO^\uparrow(1, 3) = L^\uparrow_+$: the proper orthochronus Lorentz group}.
    
\end{frame}

\begin{frame}{$Spin^c(V)$} % % % % % % % % % % % % % % % % % % %

    \begin{Definition} $\defn{\spincv} := \set{u_1 \cdots u_{2k}: u_i \in V^\CC, u^*u = 1} \subset \cclv$
    \end{Definition}
    
    
    \begin{proposition}
    The map $Ad|_{\spincv}: \spincv \to SO(V, g)$ is surjective.
    
    For any element $u \in \spincv$, both all of $\TT u$ map to the same rotation.
    \end{proposition}
    
    $Spin(V)$ is clearly a subgroup. Furthermore,  defining $\nu: \spinc \to \TT$, $u \mapsto u^! u$, we may identify $ker\,\nu = \spinv$, because that implies that $u^*= u^!$, so $u = \conj u$.
    
    $\spincv \cong \spinv \times \TT / \{\pm 1\}$, via the surjective group homomorphism $(Ad|_{\spincv}, \nu): \spincv \to \spinv \times \TT$.
    
\end{frame}

%%%%%%%%%%%%%%%%%%%%%%%%%%%%%%%%%%%%%%%%%%%%%%%%%%%%%%%%%%%%%%%%%%%%%%%%%%%%%%%
\subsection{Spinor Space / Fermionic Fock Space}

\begin{frame}{Complex Structure $J$ and $V$ as a complex vector space} % % % % % % % % % % % % % % % % % % %
    
    Suppose that $V$ be even dimensional%, so $V \cong \CC^m$, but not in a canonical way
    . A complex structure $J$ is what allows the identification of $V \cong_J \CC^m$ as complex vector spaces.
    
    \begin{definition}
    An operator $\defn{J} \in End_\RR(V)$ is an \emph{orthogonal complex structure on V} if 
        
        \begin{itemize}
            
        \item $J^2 = -1$
        
        \item $g(Ju, Jv) = g(u, v)$ for all $u, v \in V$
            
        \end{itemize}
    \end{definition}
    
    This allows to think of $V$ as an $m$ dimensional complex vector space where $i \cdot v := Jv \in V$.
    
    \emph{Remark}: This multiplication of $v$ by $i$ is different to the multiplication by $i$ in $V^\CC = V \oplus iV$, because the latter takes us outside $V$ into $iV$: \textbf{$V$ is NOT a complex v. subspace of $V^\CC$}.

\end{frame}

\begin{frame}{Inner Product and Example: $\RR^{2m}$} % % % % % % % % % % % % % % % % % % %
    
    Having chosen a complex structure $J$, we may define the \emph{sesquilinear} inner product \[ \defn{\bra} u, v \defn{\ket_J} := g(u, v) + ig(Ju, v) \]
    
    With this complex inner product, we will denote by \defn{$V_J$} the $m$-dimensional \textbf{(complex) Hilbert space} $(V, \bra \cdot, \cdot \ket_J)$.\\------------------
    
    In $\RR^{2m}$, usually 
    $J := \begin{pmatrix} 
        I_n & 0_n \\
        0_n & -I_n
    \end{pmatrix}$. In $\RR^2$ this means that $i \cdot \begin{pmatrix} x & y \end{pmatrix}^T := \begin{pmatrix} -y & x\end{pmatrix}^T $, so we are making the natural identification of $\RR^2 \ni \begin{pmatrix} x & y \end{pmatrix}^T \mapsto z = x + iy \in \CC^1$ as complex vector spaces. Notice that $\bra e_j, e_{j+m} \ket_J = i$.
    
    
    %(Do not talk about $W_J$ here, but in the section of Polarized Fock Space. Con: can't talk of the action of $w \in W_J$, which is used to prove that the action is self adjoint; can't proof important things)... Creo que mejor si, para poder probar al menos que la rep. en el caso par es irrep, y autoadjunta: $W_J$, $P_J$ 
    
\end{frame}

\begin{frame}{$W_J$: polarization of $V^\CC$} % % % % % % % % % % % % % % % % % % %
    For our purposes, this space will be useful to prove things about representations easily.
    
    $J$ is an operator $:V \to V$ in contrast to the ``operator'' $i: V\subset V^\CC \to iV \subset V^\CC$ and $i:iV \to V$.
    
    \begin{definition}
    Given an orthogonal complex structure $J$ on $V$, define $\defn{P_J}: V_J \to V^\CC$ by $v \mapsto \frac{1}{2}(1 - iJ)v = \frac{1}{2}(v - \frac{1}{2}iJv)$. We will usually denote by $\defn{w_z} := P_J(v) \in W_J$.\\
    Define $\defn{W_J} \subset V^\CC$ as the image of $P_J$. We can see that $W_J = \set{v - iJv \,|\, v \in V}$. It is a \textbf{complex vector subspace of $V^\CC$}.\\
    Define $\defn{\conj{W_J}} \subset V^\CC$ as the image of $P_{-J}$. We see that $\conj{W_J} = \set{v + iJv \,|\, v \in V}$.
    \end{definition}
    
    Notice that $V^\CC = W_J \oplus \conj{W_J}$, e.g. $V \ni v = \frac{1}{2}(w_v  + \conj{w_v})$.
    
\end{frame}

\begin{frame}{Hilbert space structure of $V^\CC$} % % % % % % % % % % % % % % % % % % %
    
    In general, we can make $V^\CC$ a (complex) Hilbert space by ``twisting'' $g^\CC$ \[ \defn{\bbra} \alpha, \beta \defn{\kket} := 2 g^\CC(\conj{\alpha}, \beta) \qquad \textbf{for } \alpha, \beta \in V^\CC \]
    
    $W_J$ and $\conj{W_J}$ are orthogonal with respect to this product, so we say that $W_J$ is a \defn{polarization of $V^\CC$}.
    
    When we restrict $\bbra \cdot , \kket$ to $W_J$ \[P_J: (V_J, \bra \cdot, \cdot \ket_J) \to (W_J, \bbra \cdot, \cdot \kket) \] \textbf{is a unitary ($\bra u, v \ket_J = \bbra w_u, w_v \kket$) isomorphism}.
    
\end{frame}

\begin{frame}{Example: $W_J$ for $\RR^4$} % % % % % % % % % % % % % % % % % % %

    In $\RR^4$, recall that, e.g. $J(e_1) = e_3$ and $J(e_3) = -e_1$.
    \begin{align*}
        W_J &= span_\CC\set{e_1 -i e_3, e_2 - ie_4} \subset{\RR^4 \otimes \CC} \\
        P_J(e_3) &= ie_1 + 1e_3 \in W_J \\
        P_J(x e_1 + y e_2 + z e_3) &= (x + iz)e_1 + (y)e_2 + (z - ix)e_3 + (-iy)e_4\in W_J
    \end{align*}
    


Notice that $i(e_1 - i e_3) = e_3 + ie_1 = e_3 - i (-e_1) = e_3 - i (J e_3)$, so $W_J$ is indeed a complex vector subspace unlike $V$.

\end{frame}

%%%%%%%%%%%%%%%%%%%%%%%%%%%%%%%%%%%%%%%
\subsubsection{Representation of $\CC l(V)$: $V$ even-dimensional}

\begin{frame}{Fock Space} % % % % % % % % % % % % % % % % % % %

    Let $V$ be an even dimensional real vector spaces of dimension $n = 2m$, and fix an orthogonal complex structure $J$ with respect to its positive-definite inner product $g$.
    
    Let $\extv_J$ be the (complex) exterior algebra over the (complex) vector space $V_J$.
    
    On $\extv_J$, induce the complex inner product $\defn{\bra} u_1 \wedge \cdots \wedge u_k, v_1 \wedge \cdots \wedge v_l \defn{\ket}_J := \delta_{kl} det[\bra u_i, v_j \ket_J]$.
    
    \begin{definition}
    Let \defn{the Fock Space} $\defn{\mathcal F_J(V)} := \extv_J$. It is a (complex) Hilbert space for the inner product $\bra \cdot, \cdot \ket_J$. Its elements are called \defn{spinors}. Choose a unit vector $\defn{\Omega} \in \CC \subset \fjv$ and call it \defn{the vacuum vector of \fjv}.
    \end{definition}

\end{frame}

\begin{frame}{Polarized Fock Space} % % % % % % % % % % % % % % % % % % %

    On $\extwj$, induce the complex inner product $\defn{\bbra} z_1 \wedge \cdots \wedge z_k, w_1 \wedge \cdots \wedge w_l \defn{\kket} := \delta_{kl} det[\bbra z_i, w_j \kket]$.
    
    \begin{definition}
    Let $\extwj$ be \defn{the Polarized Fock Space asociated to J}. It is a (complex) Hilbert space for the inner product $\bbra \cdot, \cdot \kket$. Its elements are also called \defn{spinors}. $\defn{\Omega} \in \CC \subset \fjv$ is called a \defn{the vacuum vector of \fjv}.
    \end{definition}
    
    \textbf{The Fock space and the polarized Fock space are unitarily isomorphic} via the transformation $P_J : \extvj \to \extwj$ induced by $P_J:V_J \to W_J$, given by $P_J(v_1 \wedge \cdots \wedge v_k) = P_Jv_1 \wedge \cdots \wedge P_Jv_k$.
    
\end{frame}

\begin{frame}{Example: $\mathcal F_J(\RR^{2m})$} % % % % % % % % % % % % % % % % % % %
    
    Given the standard complex structure of $\RR^{2m}$, as a complex vector space the set $\set{e_1, \dots, e_{\textbf{m}}}$ is an orthonormal basis, so $dim_\CC\RR^{2m} = m$ and $dim_\CC \bigwedge^\bullet \RR^{2m} = 2^m$, because here $i\cdot e_j = e_{j+m}$.
    
    So, a generic element of $\mathcal F(\RR^{2m})$ is: \quad $z_0 + z_{1,1}e_1 + \cdots + z_{1,m} e_m + z_{2, 12} e_1\wedge e_2 + \cdots z_{2, m-1\, m} e_{m-1} \wedge e_{m} + \cdots + z_{m, 1\cdots m} e_1 \wedge e_2 \wedge \cdots \wedge e_m $
    
    For instance, \[(1+i) (3-i)e_1 \wedge e_2 = (1+i) - i(3-i) e_1 \wedge e_4 \in \mathcal F(\RR^4)\]
    
\end{frame}

\begin{frame}{Representations of $\cclv$ \Iff Fock space}

    We can think a representation of $\cclv$ in Fock space as saying that we can multiply elements of $\cclv$ by spinors to obtain other spinors. 
    
    It is this multiplication what will formalize the $\slashed D = \gamma^\mu ( \partial_\mu + i \mu A_\mu)$ operator in the Dirac equation.
    
    Furthermore, the way $\spinv \subset \cclv$ acts on spinors, in the even dimensional case (e.g. Minkowski) will tell us that Fock space is divided in $2$: left handed spinors, and right handed spinors.
    
\end{frame}

\begin{frame}{Representation of $\cclv$ in \fjv} % % % % % % % % % % % % % % % % % % %
    
    For $v \in V$, let \defn{the creation operator $a^\dagger_J($}v\defn{$)$} be:
    \[ a^\dagger(v) (u_1 \wedge \cdots \wedge u_k) := v \wedge u_1 \wedge \cdots \wedge u_k \]
    
    And \defn{the annihilation operator $a_J($}v\defn{$)$} by:
    
    \[ a_J(v) (u_1 \wedge \cdots \wedge u_k) := \sum_{j = 1}^k \bra v, u_j \ket_J \wedge u_1 \wedge \cdots \wedge \hat u_j \wedge \cdots  \wedge u_k \]
    
    \defn{The Clifford action} of $\cclv$ on the Fock space is induced by the action of $v \in V$, which is defined by:
    
    \[ \defn{\pi_J(}v\defn{)} := a_J(v) + a_J^\dagger(v) \]

\end{frame}

\begin{frame}{Representation of $\cclv$ in polarized \fjv} % % % % % % % % % % % % % % % % % % %
    
    For $v \in V$, let \defn{the creation operator $a^\dagger_J($}v\defn{$)$} be:
    \[ a^\dagger(v) (u_1 \wedge \cdots \wedge u_k) := v \wedge u_1 \wedge \cdots \wedge u_k \]
    
    And \defn{the annihilation operator $a_J($}v\defn{$)$} by:
    
    \[ a_J(v) (u_1 \wedge \cdots \wedge u_k) := \sum_{j = 1}^k  (-1)^{j-1}\bra v, u_j \ket_J\, u_1 \wedge \cdots \wedge \hat u_j \wedge \cdots  \wedge u_k \]
    
    \defn{The Clifford action} of $\cclv$ on the Fock space is induced by the action of $v \in V$, which is defined by:
    
    \[ \defn{\pi_J(}v\defn{)} := a_J(v) + a_J^\dagger(v) \]
    
    We will see that this is the unique irreducible representation of $\cclv$.
    
\end{frame}

\begin{frame}{Theorems about Fock Space: even dimension} % % % % % % % % % % % % % % % % % % %
    \begin{theorem} It is the \textbf{unique} \emph{irreducible} algebra representation of $\CC l(V)$.\end{theorem}% irrep part comes immediately from Schur's lemma

    \begin{theorem}$\pi_J$ is a selfadjoint ($\pi_J(a^*) = \pi_J(a)^\dagger$) representation of the *-algebra $\CC l(V)$.\end{theorem}%: comes ``easily'' from knowing what the actio of $w \in W_J$ is. WE MIGHT AS WELL SAY SPINORS ARE THOSE THINGS THAT ARE ACTED ON BY SPIN/CLIFFORD.
    
    \begin{theorem}
    When the action of $\spinv$ and $\spincv$, $\mathcal F$ decomposes into irreducible representations $\mathcal F = \mathcal F^+ \oplus \mathcal F^-$. \textbf{i.e. there are $2$ types of spinors: ``left handed'' and ``right handed''}. \tiny {In our view of Fock space $\mathcal F^+ = \bigwedge^{even}V_J$, $\mathcal F^- = \bigwedge^{odd}V_J$. }
    \end{theorem}
    
    \begin{proposition} $\gamma$ acts as a $\ZZ_2$-grading operator on the Fock space. So,  $\pi_J(\gamma) \mathcal F^+ = \mathcal F^+$ and $\pi_J(\gamma) \mathcal F^- = -\mathcal F^-$. \end{proposition}%, which is the exterior algebra grading. ... useful, perhaps, because of teoremita de orden normal? -> \textbf{2 types of spinors} (which act differently by Spin, so act differently by SL( . , C)?)

\end{frame}

\begin{frame}{Example: Multiplication in $\mathcal F(\RR^2)$ and Decomposition} % % % % % % % % % % % % % % % % % % %
    We can see that $\mathcal F(\RR^2) = \set{z_1 + z_2 e_1}$, because $\set{e_1}$ alone is a basis for $\RR^2$ as a complex v. space.
    
    It is important to notice that $\pi_J(e_2) \neq i \pi_J(e_1)$, because we are considering $e_1, e_2$ as elements of $V^\CC \subset \CC l(\RR^2)$, so $e_2 \neq i e_1$.
    
    To determine the action of $\CC l(\RR^2)$ it is enough to determine the action of the (real basis) $\set{e_1, e_2}$.
    
    
    $\pi_J(e_1) (z_1) = z_1 e_1$\\
    $\pi_J(e_1)(z_2 e_1) = z_2 e_1 \wedge e_1 + z_2 \bra e_1, e_1\ket_J = z_2$.
    
    
    $\pi_J(e_2) (z_1) = z_1 e_2 = -i z_1 e_1$ \\
    $\pi_J(e_2)(z_2 e_1) = z_2 e_2 \wedge e_1 + z_2 \bra e_2, e_1\ket_J = i z_2 e_1 \wedge e_1 \bra i e_1, e_1 \ket_J = -i z_2$.
    
\end{frame}

\begin{frame}{Example(Cont.): Multiplication in $\mathcal F(\RR^2)$ and Decomposition} % % % % % % % % % % % % % % % % % % %
    Now, $\gamma = -i e_1 e_2 \in \CC l(\RR^2)$.
    
    So, $\pi_J(\gamma) (z_1 + z_2 e_1) = -i \pi_J(e_1)(-iz_2 + iz_1e_1) = z_1 - z_2 e_1$.
    
   This means that $z_1 \in \mathcal F ^+ = \bigwedge^0 \RR^2$, y para $z_2 e_1 \in \mathcal F ^- = \bigwedge^1 \RR^2$:
    
    $\pi_J(\gamma)(z_1) = z_1$ \quad and for \\
    $\pi_J(\gamma)(z_2 e_1) = -z_2 e_1$.
    
    Therefore $\frac{1}{2}(1 \pm \gamma)$ are orthogonal projectors into $\mathcal F^\pm$:
    
    $\pi_J(\frac{1}{2}(1+\gamma))(z_1 + z_2e_1) = z_1 \in \mathcal F^+$ \quad and\\
    $\pi_J(\frac{1}{2}(1-\gamma))(z_1 + z_2e_1) = z_2 e_1 \in \mathcal F^-$.

\end{frame}

\begin{frame}{Example(Cont.): Multiplication in $\mathcal F(\RR^2)$ and Decomposition} % % % % % % % % % % % % % % % % % % %
    Finally, we see that $Spin(2) = \set{x + y e_1 e_2 \,|\, x, y \in \RR; x^2 + y^2 = 1}$ acts irreducibly in $\mathcal F^+$ and $\mathcal F^-$.
    
    Notice that $\pi_J(e_1 e_2)(z_1 + z_2 e_1) = iz_1 -iz_2 e_1)$.
    
    $\mathcal F^+$:
        \[ \pi_J(x + y e_1 e_2)(z_1) = xz_1 + iyz_2 \in \mathcal F^+\]
        
    $\mathcal F^-$:
        \[ \pi_J(x + y e_1 e_2)(z_2e_1) = (xz_2 - iyz_2)e_1 \in \mathcal F^-\]

\end{frame}

\begin{frame}{Example: $\mathcal F(\RR^{1, 3})$} % % % % % % % % % % % % % % % % % % %
    
    Similarly, for $\RR^4$ with the metric of signature $(+, -, -, -)$, the spin group $\spin^\uparrow(\RR^{1, 3}) \cong SL(2, \CC)$ is the double cover of $SO^\uparrow(1, 3) = L^\uparrow_+$
    
\end{frame}

%%%%%%%%%%%%%%%%%%%%%%%%%%%%%%%%%%%%%%%
\subsubsection{Representation of $\CC l(V)$: $V$ odd-dimensional}

\begin{frame}{Fock Space: odd dimension} % % % % % % % % % % % % % % % % % % %
    As $V$ is odd dimensional, it can not have a complex structure, so it is not clear what Fock space is.
    
    Let $U = span_\RR\set{e_1, \dots, e_{2m}} \subset V$. \defn{The Fock space} will be a fock space of $\mathcal F_J(U)$.
    
    $\CC l(U) \cong \cclv^+$ via the algebra isomorphism extended from $u \in U \mapsto iue_{2m+1}$, so that $\mathcal F_J(U)$ is an irreducible rep. of $\cclv^+$.
    
    The algebra $\cclv$ is \emph{generated} by $\cclv^+$ together with the central element $\gamma$, in the sense that any $a \in \cclv = b_1 + \gamma b_2$ with $b \in \cclv^+$, because $\gamma^2 = 1$ and $\gamma$ is of odd degree.
    

\end{frame}

\begin{frame} % % % % % % % % % % % % % % % % % % %
    So, the action of all of $\cclv$ can be extended in exactly $2$, \emph{irreducible}, ways: by letting $\gamma$ act as either $+1$ or $-1$. 
    
    This representations must be the \textbf{unique, irreducible} inequivalent representations of $\cclv$.    
    
    Although there are $2$ inequivalent irreducible representations of the algebra $\CC l(V)$ which can both be seen in the same Fock space, \textbf{$Spin^\CC(V)$ and $Spin(V)$ act equivalently and irreducibly in all of Fock space}.
    
\end{frame}

\begin{frame}{Summary of Representations of $\cclv$} % % % % % % % % % % % % % % % % % % %
    Given a vector space $V$, with a metric of any signature, the Fock space $\mathcal F(V)$ can be seen as the space where all the irreducible representations of $\cclv$ act.

    In general: $dim_\CC (\mathcal F(V)) = 2^{\lfloor\frac{dim_\RR V}{2} \rfloor}$
    
    In the even dimensional case, like that of Minkowski spacetime, the irreducible representation is unique and it decomposes into $2$ irreducible representations of $\spinv$ (or of $\spin^\uparrow(V, g)$ in the general signature case), giving ``left handed'' and ``right handed'' spinors. %I believe that's the case, because I believe that spinc doesnt depend on the signature of the metric, and that Spin(V, g) is subspace of Spin^c(V).
\end{frame}