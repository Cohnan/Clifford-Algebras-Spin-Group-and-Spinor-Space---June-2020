\section{Spinors}

%%%%%%%%%%%%%%%%%%%%%%%%%%%%%%%%%%%%%%%%%%%%%%%%%%%%%%%%%%%%%%%%%%%%%%%%%%%%%%%
\subsection{Spin Groups}

\begin{frame}{From now on\dots} % % % % % % % % % % % % % % % % % % %

    Let $n = dim_\RR V$, and $m = \lfloor \frac{n}{2} \rceil$.

    From now on we assume that \emph{$g$ is positive definite}, so that $V \cong \RR^n$ as inner-product vector spaces, and so we may write $v_1 \cdot v_2 := g(v_1, v_2)$ for $v_i \in V$. %(even in the big book Elements of Noncommutative Geometry this is assumed). 
    
    Let $\set{e_1, \dots, e_n}$ be an \emph{oriented} orthonormal basis.
    
    $\gamma = (-1)^m e_1 \cdots e_n$, but it is independent of the chosen oriented orthonormal basis. % In \RR^n: that the determinant of the matrix of columns is 1, not -1.

    We will concentrate on the \textbf{complex} Clifford algebra $\CC l(V)$, with hermitian inner product $\bra \cdot , \cdot \ket$, so we will simply call it simply \emph{the Clifford algebra}.% Even if g is not positive definite, it is ISOMORPHIC, but not equal to the positive-definite case.
    
\end{frame}

\begin{frame}{Unit Vectors as Perpendicular Reflections} % % % % % % % % % % % % % % % % % % %
    Let \defn{$\cclv^\times$} be the group of invertible elements.
        \begin{itemize}
            
        \item Notice that, if $v \in V$, by the defining property of the Clifford algebra, $v^2 = ||v||^2$, we have that $v^\inv = \frac{v}{||v||^2}$.
            
        \end{itemize}
    
    Call an element $u \in \cclv$ \defn{unitary} if $u^* u = 1$. They are such that $\bra a, b \ket = \tau(a^* b) = \tau(a^*(u^* u) b)  = \bra ua, ub \ket$.
        \begin{itemize}
            
        \item If $x \in V$ is of unit length,  $x^* = \conj{(x^t)} = x = x^
        \inv$.
        
        \item Unitary vectors $w$ in $V^\CC$ \Iff $w = \lambda x$, where $\lambda \in U(1) =: \defn{\TT}$, $x$ is a unit vector in $V$.
        
        \item If $u$ is unitary \then $u \in \cclv^\times$.
            
        \end{itemize}
    

\end{frame}

\begin{frame} % % % % % % % % % % % % % % % % % % %

    For $a \in \cclv^\times$, let \defn{$Ad_a$} be the map $b \mapsto a b a^\inv$ for $b \in \cclv$. 
        \begin{itemize}
            
        \item This is a group representation, i.e. $Ad_{aa'} = Ad_a \circ Ad_{a'} \in GL(\cclv)$ 
            
        \end{itemize}

    \begin{proposition}
    For $x \in $ of unit length and $v \in V$, $Ad_x(v) = v - 2 (x \cdot v) = $ ``\textbf{Reflection} of $v$ with respect to to the plane orthogonal to $x$''. So, $Ad_x \in O(V, g)$ when applied to $V\subset \cclv$.
    \end{proposition}
    \begin{proof} Hola
    \end{proof}
    
    Finally, it is not hard to see that \textbf{applying $2$ / an even amount of reflections is a rotation on $V$}; \\the converse is also true: every rotation of $V$ may be written as an even product of reflections.
    
\end{frame}

\begin{frame}{$Spin(V)$} % % % % % % % % % % % % % % % % % % %

    Definition. Inside $Cl(V)$
    \begin{Definition} $\defn{\spinv} := \set{x_1 \cdots x_{2k}: x_i \in V, ||x_i|| = 1}$
    \end{Definition}
    
    Double Covering
    
    Examples:
        \begin{itemize}
            
        \item Sanchez, explicit
        
        \item $Spin(3) \cong SU(2)$ = {unit quaternions}
        
        \item $Spin(4) \cong SU(2) \times SU(2)$
            
        \end{itemize}
\end{frame}

\begin{frame}{$Spin^c(V)$} % % % % % % % % % % % % % % % % % % %

    Definition. Inside $Cl(V^\CC)$
    
    Exact sequence
    
    $Spin(V)$ is subgroup
    
    Examples:
        \begin{itemize}
            
        \item No se
            
        \end{itemize}

\end{frame}

%%%%%%%%%%%%%%%%%%%%%%%%%%%%%%%%%%%%%%%%%%%%%%%%%%%%%%%%%%%%%%%%%%%%%%%%%%%%%%%
\subsection{Spinor Space / Fermionic Fock Space}

\begin{frame}{Complex Structure $J$ and $V$ as a complex vector space} % % % % % % % % % % % % % % % % % % %

    Inner product.
    
    Example: $R^{2n}$
    
    (Do not talk about $W_J$ here, but in the section of Polarized Fock Space. Con: can't talk of the action of $w \in W_J$, which is used to proof that the action is self adjoint; can't proof important things)... Creo que mejor si, para poder probar al menos que la rep. en el caso par es irrep, y autoadjunta: $W_J$, $P_J$ 

\end{frame}
%%%%%%%%%%%%%%%%%%%%%%%%%%%%%%%%%%%%%%%
\subsubsection{Representation of $\CC l(V)$: $V$ even-dimensional}

\begin{frame}{Fock and Polarized Fock Space: even dimension} % % % % % % % % % % % % % % % % % % %

    Definitions
    
    Inner product

\end{frame}

\begin{frame}{Theorems about Fock Space: even dimension} % % % % % % % % % % % % % % % % % % %

    Theorem: $c_J$ is a selfadjoint representation of the *-algebra $\CC l(V)$: comes ``easily'' from knowing what the actio of $w \in W_J$ is.
    
    Theorem (Varilly chiquito): It is the \textbf{unique} \emph{irreducible} algebra representation of $\CC l(V)$: irrep comes immediately from Schur's lemma
    
    Lemma: $\gamma$ acts as the grading operator on the Fock space, which is the exterior algebra grading. ... useful, perhaps, because of teoremita de orden normal? -> \textbf{2 types of spinors} (which act differently by Spin, so act differently by SL( . , C)?)

\end{frame}

\begin{frame}{Example of Fock Space} % % % % % % % % % % % % % % % % % % %

    Example: Sanchez. Chirality is what we expect.

\end{frame}

%%%%%%%%%%%%%%%%%%%%%%%%%%%%%%%%%%%%%%%
\subsubsection{Representation of $\CC l(V)$: $V$ odd-dimensional}

\begin{frame}{Fock Space: odd dimension} % % % % % % % % % % % % % % % % % % %

    Paragraph of Varilly grande.
    
    Many possibilities for the FOck space, as there are many ``orthonormal hyperplanes''
    
    In general: $dim_\CC (\mathcal F(V)) = 2^{\lfloor\frac{dim_\RR V}{2} \rfloor}$
    
    Theorem: 
        \begin{itemize}
        
        \item There are exactly 2 unitarily inequivalent irreps, both can be seen on Fock space. (Varilly chiquito)
        
        \item The 2 irreps agree in the even part of the algebra.
        
        \end{itemize}

\end{frame}

%%%%%%%%%%%%%%%%%%%%%%%%%%%%%%%%%%%%%%%
\subsubsection{Spin Groups Representations}

\begin{frame}{Results} % % % % % % % % % % % % % % % % % % %

    Even:
        \begin{itemize}
            
        \item $Spin^C$ and $Spin$ act irreducibly and inequivalently on the positive (chirality) and negative (chirality) of Fock space. \textbf{i.e. there are 2 types of spinors, each of which acts differently under Spin, so under Lorentz $\subset SL(, \CC)$}!!
            
        \end{itemize}
    
    Odd:
        \begin{itemize}
        
        \item Although there are $2$ inequivalent irreps of the algebra $\CC l(V)$ which can both be seen in the same Fock space, $Spin^\CC$ and $Spin$ act equivalently and irreducibly in this Fock space. TODO: que significa?
        
        \end{itemize}

\end{frame}