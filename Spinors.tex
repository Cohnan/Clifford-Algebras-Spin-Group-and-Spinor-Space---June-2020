\section{Spinors}

%%%%%%%%%%%%%%%%%%%%%%%%%%%%%%%%%%%%%%%%%%%%%%%%%%%%%%%%%%%%%%%%%%%%%%%%%%%%%%%
\subsection{Spin Groups}

\begin{frame}{From now on\dots} % % % % % % % % % % % % % % % % % % %

    Let $n = dim_\RR V$, and $m = \lfloor \frac{n}{2} \rceil$.

    From now on we assume that \emph{$g$ is positive definite}, so that $V \cong \RR^n$ as inner-product vector spaces, and so we may write $v_1 \cdot v_2 := g(v_1, v_2)$ for $v_i \in V$. %(even in the big book Elements of Noncommutative Geometry this is assumed). 
    
    Let $\set{e_1, \dots, e_n}$ be an \emph{oriented} orthonormal basis.
    
    $\gamma = (-1)^m e_1 \cdots e_n$, but it is independent of the chosen oriented orthonormal basis. % In \RR^n: that the determinant of the matrix of columns is 1, not -1.

    We will concentrate on the \textbf{complex} Clifford algebra $\CC l(V)$, with hermitian inner product $\bra \cdot , \cdot \ket$, so we will simply call it simply \emph{the Clifford algebra}.% Even if g is not positive definite, it is ISOMORPHIC, but not equal to the positive-definite case.
    
\end{frame}

\begin{frame}{Unit Vectors as Perpendicular Reflections} % % % % % % % % % % % % % % % % % % %
    Let \defn{$\cclv^\times$} be the group of invertible elements.
        \begin{itemize}
            
        \item Notice that, if $v \in V$, by the defining property of the Clifford algebra, $v^2 = ||v||^2$, we have that $v^\inv = \frac{v}{||v||^2}$.
            
        \end{itemize}
    
    Call an element $u \in \cclv$ \defn{unitary} if $u^* u = 1$. They are such that $\bra a, b \ket = \tau(a^* b) = \tau(a^*(u^* u) b)  = \bra ua, ub \ket$.
        \begin{itemize}
            
        \item If $x \in V$ is of unit length,  $x^* = \conj{(x^t)} = x = x^
        \inv$.
        
        \item Unitary vectors $w$ in $V^\CC$ \Iff $w = \lambda x$, where $\lambda \in U(1) =: \defn{\TT}$, $x$ is a unit vector in $V$.
        
        \item If $u$ is unitary \then $u \in \cclv^\times$.
            
        \end{itemize}
    

\end{frame}

\begin{frame} % % % % % % % % % % % % % % % % % % %

    For $a \in \cclv^\times$, let \defn{$Ad_a$} be the map $b \mapsto a b a^\inv$ for $b \in \cclv$. 
        \begin{itemize}
            
        \item This is a group representation, i.e. $Ad_{aa'} = Ad_a \circ Ad_{a'} \in GL(\cclv)$ 
            
        \end{itemize}

    \begin{lemma}
    For $x \in $ of unit length and $v \in V$, $Ad_x(v) = v - 2 (x \cdot v) = $ ``\textbf{Reflection} of $v$ with respect to to the plane orthogonal to $x$''. So, $Ad_x \in O(V, g)$ when applied to $V\subset \cclv$.
    \end{lemma}
    \begin{proof} Hola
    \end{proof}
    
    Finally, it is not hard to see that \textbf{applying $2$ / an even amount of reflections is a rotation: $Ad_{x_1 x_2}\in SO(V, g)$}; \\the converse is also true: every rotation of $V$ may be written as an even product of reflections.
    
\end{frame}

\begin{frame}{$Spin(V)$} % % % % % % % % % % % % % % % % % % %

    \begin{Definition} $\defn{\spinv} := \set{x_1 \cdots x_{2k}: x_i \in V, ||x_i||^2 = x_i^* x_i = 1} \subset \clvg \subset \cclv$
    \end{Definition}
    
    We saw that $\spinv$ acts on $V$, by the adjoint representation, by rotations. Furthermore:
    
    \begin{proposition}
    The map $Ad|_{\spinv}: \spinv \to SO(V, g)$ is surjective.
    
    For any element $u \in \spinv$, both and $u, -u$ map to the same rotation.
    \end{proposition}
    
    TODO: it may be charactezed by $u \in \cclv$ is in $\spinv$ \iff $u$ is even, unitary and real ($u = \conj u$).
\end{frame}

\begin{frame}{Example: $\spin(2)$} % % % % % % % % % % % % % % % % % % %
    
    To see: \[\spin(2) = \set{x + y e_1 e_2 \,|\, x, y \in \RR;  x^2 + y^2 = 1} \cong \TT\]
    
    $\spin(2) \subset \set{x + y e_1 e_2 \,|\, x, y \in \RR;  x^2 + y^2 = 1}$:
        \begin{itemize}
            
        \item Product of $2$ unit vectors $\subset \set{x + y e_1 e_2}$: Let $u_1 = a e_1 + b e_2$ and $u_2 = \alpha e_1 + \beta e_2$ be normalized vectors in $\RR^2$. Then $u_1 u_2 = x + e_1 e_2 y \in \spin(2)$, with $x = a \alpha + b \beta$ and $y = a \beta  - b \alpha$ and $x^2 + y^2 = (a^2 + b^2)(\alpha^2 + \beta^2) = 1$.
        
        \item Even product of unit vectors $\subset$ Product of elements of $\set {x + y e_1 e_2} \subset \set {x + y e_1 e_2}$: $(x_1 + y_1 e_1 e_2)(x_2 + y_1 e_1 e_2) = x + y e_1 e_2$ with $x = x_1 y_1 - x_2 y_2$ and $y = x_1 y_2 + x_2 y_1$ and so $x^2 + y^2 = 1$.
            
        \end{itemize}
    
\end{frame}

\begin{frame}{Example (Cont.): $\spin(2)$} % % % % % % % % % % % % % % % % % % %
    
    (Cont.) To see: \[\spin(2) = \set{x + y e_1 e_2 \,|\, x, y \in \RR;  x^2 + y^2 = 1} \cong \TT\]
    
    $\spin(2) \subset \set{x + y e_1 e_2 \,|\, x, y \in \RR;  x^2 + y^2 = 1}$:
        
        \begin{itemize}
        
        \item Any element $x + y e_1 e_2$ can be seen as the product of unit vectors $e_1$ and $x e_1 + y e_2$: $e_1 (x e_1 + y e_2) = x e_1^2 + y e_1 e_2 = x + y e_1 e_2 \in \spin(2)$
            
        \end{itemize}
    \qed
    
    Looking at $\spin(2)$ as a ``subset'' of $M_2(\RR) \cong Cl(2)$, then $\spin(2) \cong \set{\begin{pmatrix} x & -y \\ y & x \end{pmatrix} \,|\, x^2 + y^2 = 1} \subset M_2(\RR)$.
    
\end{frame}

\begin{frame}{Example: $SU(3)$} % % % % % % % % % % % % % % % % % % %

    The Lie algebra of $Spin(3) \subset Cl(\RR^3)$ is $Q(\bigwedge^\bullet \RR^3) = Cl^0(\RR^3) = \HH$, where $\hat i = e_2 e_3, \hat j = e_3 e_1, \hat k = e_1 e_2$.
    
    $Cl(\RR^3) \cong M_2(\CC)$ via the identification $e_j \mapsto \sigma_j$; so $\hat i \mapsto i \sigma_1, \hat j \mapsto i \sigma_2, \hat i \mapsto k \sigma_3$.
    
    The Lie algebra of $SU(2)$ is precisely that generated by $i\sigma_1, i\sigma_2, i\sigma_3$: they satisfy the commutation relations $[i\sigma_i, i \sigma_j] = 2 \epsilon_{ijk} (i \sigma_k)$; every element of $SU(2)$ may be written as $exp\{\frac{1}{2} \phi \, \vec n \cdot \vec{i \sigma}\}$.
    
    So, under the isomorphism $\CC l(\RR^3) \cong M_2(\CC) \oplus M_2(\CC)$ the Lie algebra of $\spin(3)$ is identified with the Lie algebra of \rtext{$SU(2)$ \Then $Spin(3) \cong SU(2)$}.
    
\end{frame}

\begin{frame}{$Spin^c(V)$} % % % % % % % % % % % % % % % % % % %

    \begin{Definition} $\defn{\spincv} := \set{u_1 \cdots u_{2k}: u_i \in V^\CC, u^*u = 1} \subset \cclv$
    \end{Definition}
    
    
    \begin{proposition}
    The map $Ad|_{\spincv}: \spincv \to SO(V, g)$ is surjective.
    
    For any element $u \in \spincv$, both all of $\TT u$ map to the same rotation.
    \end{proposition}
    
    $Spin(V)$ is clearly a subgroup. Furthermore,  defining $\nu: \spinc \to \TT$, $u \mapsto u^! u$, we may identify $ker\,\nu = \spinv$, because that implies that $u^*= u^!$, so $u = \conj u$.
    
    $\spincv \cong \spinv \times \TT / \{\pm 1\}$, via the surjective group homomorphism $(Ad|_{\spincv}, \nu): \spincv \to \spinv \times \TT$.
    
\end{frame}

%%%%%%%%%%%%%%%%%%%%%%%%%%%%%%%%%%%%%%%%%%%%%%%%%%%%%%%%%%%%%%%%%%%%%%%%%%%%%%%
\subsection{Spinor Space / Fermionic Fock Space}

\begin{frame}{Complex Structure $J$ and $V$ as a complex vector space} % % % % % % % % % % % % % % % % % % %
    
    Suppose that $V$ be even dimensional%, so $V \cong \CC^m$, but not in a canonical way
    . A complex structure $J$ is what allows the identification of $V \cong_J \CC^m$ as complex vector spaces.
    
    \begin{definition}
    An operator $\defn{J} \in End_\RR(V)$ is an \emph{orthogonal complex structure on V} if 
        
        \begin{itemize}
            
        \item $J^2 = -1$
        
        \item $g(Ju, Jv) = g(u, v)$ for all $u, v \in V$
            
        \end{itemize}
    \end{definition}
    
    This allows to think of $V$ as an $m$ dimensional complex vector space where $i \cdot v := Jv \in V$.
    
    \emph{Remark}: This multiplication of $v$ by $i$ is different to the multiplication by $i$ in $V^\CC = V \oplus iV$, because the latter takes us outside $V$ into $iV$.

\end{frame}

\begin{frame}{Example: $\RR^{2m}$} % % % % % % % % % % % % % % % % % % %
    
    In $\RR^{2m}$, usually 
    $J := \begin{pmatrix} 
        I_n & 0_n \\
        0_n & -I_n
    \end{pmatrix}$.
    
    In $\RR^2$ this means that $i \twov{x}{y} := \twov{-y}{x}$, so we are making the natural identification of $\RR^2 \ni \twov{x}{y} \mapsto z = x + iy \in \CC^1$ as complex vector spaces.
    
    Having chosen a complex structure $J$, we may define the \emph{sesquilinear} inner product \[ \defn{\bra} u, v \defn{\ket_J} := g(u, v) + ig(Ju, v) \]
    
    With this complex inner product, we will denote by \defn{$V_J$} the $m$-dimensional \textbf{(complex) Hilbert space} $(V, \bra \cdot, \cdot \ket_J)$.
    
    
    %(Do not talk about $W_J$ here, but in the section of Polarized Fock Space. Con: can't talk of the action of $w \in W_J$, which is used to prove that the action is self adjoint; can't proof important things)... Creo que mejor si, para poder probar al menos que la rep. en el caso par es irrep, y autoadjunta: $W_J$, $P_J$ 
    
\end{frame}

\begin{frame}{$W_J$} % % % % % % % % % % % % % % % % % % %
    For our purposes, this space will be useful to prove things about representations easily.
    
    $J$ is an operator $:V \to V$ in contrast to the ``operator'' $i: V\subset V^\CC \to iV \subset V^\CC$ and $i:iV \to V$.
    
    \begin{definition}
    Given an orthogonal complex structure $J$ on $V$, define $\defn{P_J}: V_J \to V^\CC$ by $v \mapsto \frac{1}{2}(1 - iJ)v = \frac{1}{2}(v - \frac{1}{2}iJv)$. We will usually denote by $\defn{w_z} := P_J(v) \in W_J$.\\
    Define $\defn{W_J} \subset V^\CC$ as the image of $P_J$. We can see that $W_J = \set{v - iJv \,|\, v \in V}$. It is a \textbf{complex vector subspace of $V^\CC$}.\\
    Define $\defn{\conj{W_J}} \subset V^\CC$ as the image of $P_{-J}$. We see that $\conj{W_J} = \set{v + iJv \,|\, v \in V}$.
    \end{definition}
    
    Notice that $V^\CC = W_J \oplus \conj{W_J}$, e.g. $V \ni v = \frac{1}{2}(w_v  + \conj{w_v})$.
    
\end{frame}

\begin{frame}{Hilbert space structure of $V^\CC$} % % % % % % % % % % % % % % % % % % %
    
    In general, we can make $V^\CC$ a (complex) Hilbert space by ``twisting'' $g^\CC$ \[ \defn{\bbra} \alpha, \beta \defn{\kket} := 2 g^\CC(\conj{\alpha}, \beta) \qquad \textbf{for } \alpha, \beta \in V^\CC \]
    
    $W_J$ and $\conj{W_J}$ are orthogonal with respect to this product, so we say that $W_J$ is a \defn{polarization of $V^\CC$}.
    
    When we restrict $\bbra \cdot , \kket$ to $W_J$ \[P_J: (V_J, \bra \cdot, \cdot \ket_J) \to (W_J, \bbra \cdot, \cdot \kket) \] \textbf{is a unitary ($\bra u, v \ket_J = \bbra w_u, w_v \kket$) isomorphism}.
    
\end{frame}

\begin{frame}{Example: $\RR^4$} % % % % % % % % % % % % % % % % % % %

In $\RR^4$, recall that, e.g. $J(e_1) = e_3$ and $J(e_3) = -e_1$.
\[
    W_J = span_\CC\set{e_1 -i e_3, e_2 - ie_4} \subset{\RR^4 \otimes \CC}
\]

Notice that $i(e_1 - i e_3) = e_3 + ie_1 = e_3 - i (-e_1) = e_3 - i (J e_3)$, so $W_J$ is indeed a complex vector subspace unlike $V$.

\end{frame}

%%%%%%%%%%%%%%%%%%%%%%%%%%%%%%%%%%%%%%%
\subsubsection{Representation of $\CC l(V)$: $V$ even-dimensional}

\begin{frame}{Fock and Polarized Fock Space: even dimension} % % % % % % % % % % % % % % % % % % %

    \begin{definition}
    \end{definition}
    
    Inner product

\end{frame}

\begin{frame}{Theorems about Fock Space: even dimension} % % % % % % % % % % % % % % % % % % %

    Theorem: $c_J$ is a selfadjoint representation of the *-algebra $\CC l(V)$: comes ``easily'' from knowing what the actio of $w \in W_J$ is.
    
    Theorem (Varilly chiquito): It is the \textbf{unique} \emph{irreducible} algebra representation of $\CC l(V)$: irrep comes immediately from Schur's lemma
    
    Lemma: $\gamma$ acts as the grading operator on the Fock space, which is the exterior algebra grading. ... useful, perhaps, because of teoremita de orden normal? -> \textbf{2 types of spinors} (which act differently by Spin, so act differently by SL( . , C)?)

\end{frame}

\begin{frame}{Example of Fock Space} % % % % % % % % % % % % % % % % % % %

    Example: Sanchez. Chirality is what we expect.

\end{frame}

%%%%%%%%%%%%%%%%%%%%%%%%%%%%%%%%%%%%%%%
\subsubsection{Representation of $\CC l(V)$: $V$ odd-dimensional}

\begin{frame}{Fock Space: odd dimension} % % % % % % % % % % % % % % % % % % %

    Paragraph of Varilly grande.
    
    Many possibilities for the FOck space, as there are many ``orthonormal hyperplanes''
    
    In general: $dim_\CC (\mathcal F(V)) = 2^{\lfloor\frac{dim_\RR V}{2} \rfloor}$
    
    Theorem: 
        \begin{itemize}
        
        \item There are exactly 2 unitarily inequivalent irreps, both can be seen on Fock space. (Varilly chiquito)
        
        \item The 2 irreps agree in the even part of the algebra.
        
        \end{itemize}

\end{frame}

%%%%%%%%%%%%%%%%%%%%%%%%%%%%%%%%%%%%%%%
\subsubsection{Spin Groups Representations}

\begin{frame}{Results} % % % % % % % % % % % % % % % % % % %

    Even:
        \begin{itemize}
            
        \item $Spin^C$ and $Spin$ act irreducibly and inequivalently on the positive (chirality) and negative (chirality) of Fock space. \textbf{i.e. there are 2 types of spinors, each of which acts differently under Spin, so under Lorentz $\subset SL(, \CC)$}!!
            
        \end{itemize}
    
    Odd:
        \begin{itemize}
        
        \item Although there are $2$ inequivalent irreps of the algebra $\CC l(V)$ which can both be seen in the same Fock space, $Spin^\CC$ and $Spin$ act equivalently and irreducibly in this Fock space. TODO: que significa?
        
        \end{itemize}

\end{frame}