\section{Clifford Algebras}



\begin{frame}{Frame Title} % % % % % % % % % % % % % % % % % % %
    \tiny
    Quihubo. Que pena, le recomiendo estas cositas que son necesarias para mi parte de la exposicion:
    
        \begin{itemize}
        
        \item OJO: el mapa $\sigma : Cl(V, g) \to \bigwedge ^\bullet V$ como lo define Sanchez no es correcto (lo es solo si los vectores son todos ortogonales, de lo contrario no). La version correcta esta, por ejemplo, en el documento que le mande en la tarde.
        
        \item Dependiendo de como defina el producto interno $\bra \cdot, \cdot \ket$ en $\cclv$, ver que cuando se aplica a vectores en $V$ coincide con el producto $g$.
        
        \item El ejemplo de la tesis: 
        
            \begin{itemize}
                
            \item \tiny Identificacion de $Cl(\RR^2)$ con $M_2(\RR)$
            \end{itemize}
            
        \end{itemize}
    
    \normalsize
\end{frame}


\begin{frame}
    \tiny 
    Me pareceria chevere poner, pero usted elige:
    
        \begin{itemize}
        
        \item \textbf{Esto me pareceria muy bueno hacerlo, porque todo el tiempo se va a estar hablando del algebra exterior del espacio vectorial y muchos seguramente no tienen una intuicion de lo que eso es, asi que pueden quedar perdidos desde el principio}. El conjunto $\bigwedge^k V$ es el espacio vectorial generado por $k$-paralelepipedos orientados. Por ejemplo, 
        
            \begin{itemize}
                
            \item \tiny $\bigwedge^1 V = V$ son vectores (1-paralelepipedos orientado)
            
            \item $\bigwedge^2 V = V$ son areas orientadas. Por ejemplo en $\RR^3$ $ \threev{1}{0}{0} \wedge \threev{0}{1}{0} = e_1 \wedge e_2$ es el cuadrado (2-paralelepipedo) de lado $1$ en el plano $x-y$, y con cierta orientacion porque si primero me voy en direccion de $\threev{1}{0}{0}$ y luego de $\threev{0}{1}{0}$ entonces estoy girando en direccion ``positiva'', counterclockwise, el vector de velocidad angular va en direccion de $z$ positivo; por el contrario $e_2 \wedge e_1 = -e_1 \wedge e_2$ es tambien el cuadrado de lado $1$ en el plano $x-y$ pero con la orientacion opuesta porque si primero camino en direccionde $e_2$ y luego de $e_1$ el vector de velocidad angular va en direccion de $z$-negativo.
            
            \item $\bigwedge^0 V = \RR$ son puntos (0-paralelepipedos) con tamaño.
        
            \end{itemize}  
        
        \textbf{El producto cuña} es entonces una \textbf{generalizacion del producto cruz: el producto cruz} usual coge $2$ $1$-paralelepipedos/vectores $v, w$ y les asigna el $2$-paralelepipedo orientado / area orientada / plano orientado  $v \wedge w$ identificado por su vector perpendicular $v \times w$ cuya magnitud es igual al area ($|v||w| sin(\theta)$) con signo del paralelepipedo $v \wedge w$! Otro ejemplo ilustrativo es el producto cuña entre un vector y un area orientada es un volumen orientado, y vemos que si el vector es paralelo a esta area orientada, entonces el volumen orientado obtenido es $0$, asi como el producto cuña entre vectores paralelos es $0$.
        
        \item Entendiendo asi el producto cuña, entender a $\clvg$ como el subalgebra de $End(\extv)$ tal que \dots es lo mismo que decir que puedo multiplicar a $a \in \clvg$ por un elemento de $\extv$. En particular, \textbf{multiplicar $v \in V \subset \clvg$ al vector $w \in V$, es decir ``multiplicar'' $v$ con $w$ nos da \[vw = \epsilon(v)w + \iota(v)w = v \wedge w + v \cdot w "= v \times w + v \cdot w "\]} De hecho, esto es mas general: multiplicar a $v \in V \subset \clvg$ por $v_1 \wedge \cdots \wedge v_k$ es una version general de ``p. cu\~na/cruz + p. punto'' $v \,v_1 \wedge \cdots \wedge v_k = v \wedge v_1 \wedge \cdots \wedge v_k + meet(v_1, v_1 \wedge \cdots \wedge v_k)$\\ = $\epsilon(v) v_1 \wedge \cdots \wedge v_k + \iota(v) v_1 \wedge \cdots \wedge v_k$
        
        \item La parte par de $Cl(\RR^2)$ es isomorfa a $\CC$.
        
        \item La parte par de $Cl(\RR^3)$ es isomorfa a $\HH$ (y por muy buenas razones, casi que de aqui surgieron las algebras de Clifford: como algebras geometricas; H nacio para ser El Grupo de Rotaciones de $\RR^3$ asi como $\CC$ es el grupo de rotaciones de $\RR^2$)
        
        \item $Cl(\RR^3)$ es isomorfo como algebra a $M_2(\CC)$ donde el isomorfismo está inducido por el mapa $\RR^3 \to M_2(\CC)$, $e_1 \mapsto \sigma_i$... esto lo sabiamos/intuíamos del hecho de que las matrices de Pauli satisfacieran $\set{\sigma_i, \sigma_j} = 2 \delta_{ij}$
        
        \item $Cl(\RR^{1, 3}) = ?$... no se, solo se que su parte par es isomorfa a $Cl(\RR^3)$.
            
        \end{itemize}
\end{frame}
