\section{Clifford Algebras}



\begin{frame}{Frame Title} % % % % % % % % % % % % % % % % % % %
    \tiny
    Quihubo. Que pena, le recomiendo estas cositas que son necesarias para lo que voy a hacer:
    
        \begin{itemize}
        
        \item OJO: el mapa $\sigma : Cl(V, g) \to \bigwedge ^\bullet V$
        
        \item El ejemplo de la tesis: 
        
            \begin{itemize}
                
            \item Identificacion de $Cl(\RR^2)$ con $M_2(\RR)$
            
            \item 
            
            \item 
                
            \end{itemize}
            
        \end{itemize}
    
    Me pareceria chevere poner, pero usted elige:
    
        \begin{itemize}
            
        \item La parte par de $Cl(\RR^2)$ es isomorfa a $\CC$.
        
        \item La parte par de $Cl(\RR^3)$ es isomorfa a $\HH$ (y por muy buenas razones, casi que de aqui surgieron las algebras de Clifford: como algebras geometricas; H nacio para ser El Grupo de Rotaciones de $\RR^3$ asi como $\CC$ es el grupo de rotaciones de $\RR^2$)
        
        \item $Cl(\RR^3)$ es isomorfo como algebra a $M_2(\CC)$ donde el isomorfismo está inducido por el mapa $\RR^3 \to M_2(\CC)$, $e_1 \mapsto \sigma_i$... esto lo sabiamos/intuíamos del hecho de que las matrices de Pauli satisfacieran $\set{\sigma_i, \sigma_j} = 2 \delta_{ij}$
        
        \item $Cl(\RR^{1, 3}) = ?$... no se, solo se que su parte par es isomorfa a $Cl(\RR^3)$.
            
        \end{itemize}
    
    \normalsize
\end{frame}
