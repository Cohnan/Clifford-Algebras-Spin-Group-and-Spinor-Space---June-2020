\documentclass{beamer}

%%% Packages
\usepackage{amssymb, amsmath, amsthm}
\usepackage[english]{babel}
\usepackage[utf8]{inputenc}

%% Other Packages
\usepackage[T1]{fontenc}
\usepackage[normalem]{ulem} % For strikethrough with command \sout
\usepackage{slashed} % For slashed in Dirac operator

%%%%%% Theorem Environments
\theoremstyle{definition}

%\newtheorem{theorem}{Theorem}[section]  %numbered according to section environment, so in section to it restarts as 2.1 
\newtheorem{proposition}{Proposition}[section]  %numbered according to section environment, so in section to it restarts as 2.1 
%\newtheorem{lemma}[theorem]{section}     %numbering shared with theorem 
%\newtheorem{corollary}{Corollary}[theorem]


\theoremstyle{remark}

\newtheorem*{remark}{Remark}

%%%%% Color text
%\usepackage[]{xcolor}

\newcommand{\ytext}[1]{\textcolor{yellow}{#1}}
\newcommand{\otext}[1]{\textcolor{orange}{#1}}
\newcommand{\rtext}[1]{\textcolor{red}{#1}}
\newcommand{\lbtext}[1]{\textcolor{cyan}{#1}}
\newcommand{\dbtext}[1]{\textcolor{blue}{#1}}
\newcommand{\ptext}[1]{\textcolor{Plum}{#1}}
\newcommand{\lgtext}[1]{\textcolor{LimeGreen}{#1}}
\newcommand{\dgtext}[1]{\textcolor{OliveGreen}{#1}}

%%%%% Custom Commands
\newcommand{\set}[1]{\{#1\}}
\newcommand{\inv}{{-1}}
\newcommand{\conj}[1]{\overline{#1}}

\newcommand{\twov}[2]{\begin{pmatrix} #1 \\ #2 \end{pmatrix}}
\newcommand{\threev}[3]{\begin{pmatrix} #1 \\ #2 \\ #3 \end{pmatrix}}

\newcommand{\then}{\ensuremath{\longrightarrow}}
\newcommand{\Then}{\ensuremath{\Longrightarrow}}
\renewcommand{\iff}{\ensuremath{\longleftrightarrow}}
\newcommand{\Iff}{\ensuremath{\Longleftrightarrow}}
\newcommand{\suff}{\ensuremath{\longleftarrow}}
\newcommand{\Suff}{\ensuremath{\Longleftarrow}}


\newcommand{\bra}{\langle}
\newcommand{\ket}{\rangle}
\newcommand{\bbra}{\langle\langle}
\newcommand{\kket}{\rangle\rangle}

\newcommand{\fjv}{\ensuremath{\mathcal F_J(V)}}
%\newcommand{\fjv}{\ensuremath{\mathcal F_J(W)}}
\newcommand{\clvg}{Cl(V, g)}
\newcommand{\cclv}{\CC l(V)}
\newcommand{\spin}{\ensuremath{Spin}}
\newcommand{\spinv}{\ensuremath{\spin(V)}}
\newcommand{\spinc}{\ensuremath{Spin^\CC}}
\newcommand{\spincv}{\ensuremath{\spinc(V)}}
\newcommand{\extv}{\bigwedge^\bullet V}
\newcommand{\extvj}{\bigwedge^\bullet V_J}
\newcommand{\extwj}{\bigwedge^\bullet W_J}


\newcommand{\bb}[1]{\mathbb #1}
\newcommand{\ZZ}{\mathbb Z}
\newcommand{\RR}{\mathbb R}
\newcommand{\CC}{\mathbb C}
\newcommand{\HH}{\mathbb H}
\newcommand{\TT}{\mathbb T}

\newcommand{\defn}[1]{\lbtext{#1}}
\newcommand{\Defn}[1]{\dbtext{#1}}

%%%%%
% Choose how your presentation looks.
%
% For more themes, color themes and font themes, see:
% http://deic.uab.es/~iblanes/beamer_gallery/index_by_theme.html
%
\mode<presentation>
{
  \usetheme{Darmstadt}      % or try Darmstadt, Madrid, Warsaw, ...
  \usecolortheme{beaver}%seagull} % or try albatross, beaver, crane, ...
  \usefonttheme{default}  % or try serif, structurebold, ...
  \setbeamertemplate{navigation symbols}{}
  \setbeamertemplate{caption}[numbered]
} 

%%%%%% Show TOC before a new Section
\AtBeginSection[]
{
    \begin{frame}{Table of Contents}
        \tableofcontents[currentsection]
    \end{frame}
}

% Show TOC before a new SubSection
\AtBeginSubsection[]
{
    \begin{frame}{Table of Contents}
        \tableofcontents[currentsection,currentsubsection]
    \end{frame}
}
% Reduce size of TOC
\AtBeginDocument{
  %\addtocontents{toc}{\tiny}
  %\addtocontents{subsection in toc}{\tiny}
}

%\setbeamerfont{subsection in toc}{size=\tiny}

%%%%%% Make paragraphs start with no indentation and leave spaces between paragraphs
\setlength{\parindent}{0em}
\setlength{\parskip}{1em}

%%%%%% Document Information
\title[l]{Clifford Algebras, Spin Groups and Spinors}
\author{Juan David Prada\\ Sebastian Puerto}
\institute{Universidad de los Andes}
\date{June 26, 2020}

%%%%%%%%%%%%%%%%%%%%%%%%%%%%%%%%%%%%%%%%%%%%%%%%%%%%%%%%%%%%%%%%%%%%%%%%%%%%%%%
%%%%%%%%%%%%%%%%%%%%%%%%%%%%%%%%%%%%%%%%%%%%%%%%%%%%%%%%%%%%%%%%%%%%%%%%%%%%%%%
%%%%%%%%%%%%%%%%%%%%%%%%%%%%%%%%%%%%%%%%%%%%%%%%%%%%%%%%%%%%%%%%%%%%%%%%%%%%%%%
%%%%%%%%%%%%%%%%%%%%%%%%%%%%%%%%%%%%%%%%%%%%%%%%%%%%%%%%%%%%%%%%%%%%%%%%%%%%%%%
\begin{document}

\begin{frame}
  \titlepage
\end{frame}

% Uncomment these lines for an automatically generated outline.
\begin{frame}{Outline}
  \tableofcontents
\end{frame}

%%%%%%%%%%%%%%%%%%%%%%%%%%%%%%%%%%%%%%%%%%%%%%%%%%%%%%%%%%%%%%%%%%%%%%%%%%%%%%%
%%%%%%%%%%%%%%%%%%%%%%%%%%%%%%%%%%%%%%%%%%%%%%%%%%%%%%%%%%%%%%%%%%%%%%%%%%%%%%%
\section{Clifford Algebras}



\begin{frame}{Frame Title} % % % % % % % % % % % % % % % % % % %
    \tiny
    Quihubo. Que pena, le recomiendo estas cositas que son necesarias para mi parte de la exposicion:
    
        \begin{itemize}
        
        \item OJO: el mapa $\sigma : Cl(V, g) \to \bigwedge ^\bullet V$ como lo define Sanchez no es correcto (lo es solo si los vectores son todos ortogonales, de lo contrario no). La version correcta esta, por ejemplo, en el documento que le mande en la tarde.
        
        \item Dependiendo de como defina el producto interno $\bra \cdot, \cdot \ket$ en $\cclv$, ver que cuando se aplica a vectores en $V$ coincide con el producto $g$.
        
        \item El ejemplo de la tesis: 
        
            \begin{itemize}
                
            \item \tiny Identificacion de $Cl(\RR^2)$ con $M_2(\RR)$
            \end{itemize}
            
        \end{itemize}
    
    \normalsize
\end{frame}


\begin{frame}
    \tiny 
    Me pareceria chevere poner, pero usted elige:
    
        \begin{itemize}
        
        \item \textbf{Esto me pareceria muy bueno hacerlo, porque todo el tiempo se va a estar hablando del algebra exterior del espacio vectorial y muchos seguramente no tienen una intuicion de lo que eso es, asi que pueden quedar perdidos desde el principio}. El conjunto $\bigwedge^k V$ es el espacio vectorial generado por $k$-paralelepipedos orientados. Por ejemplo, 
        
            \begin{itemize}
                
            \item \tiny $\bigwedge^1 V = V$ son vectores (1-paralelepipedos orientado)
            
            \item $\bigwedge^2 V = V$ son areas orientadas. Por ejemplo en $\RR^3$ $ \threev{1}{0}{0} \wedge \threev{0}{1}{0} = e_1 \wedge e_2$ es el cuadrado (2-paralelepipedo) de lado $1$ en el plano $x-y$, y con cierta orientacion porque si primero me voy en direccion de $\threev{1}{0}{0}$ y luego de $\threev{0}{1}{0}$ entonces estoy girando en direccion ``positiva'', counterclockwise, el vector de velocidad angular va en direccion de $z$ positivo; por el contrario $e_2 \wedge e_1 = -e_1 \wedge e_2$ es tambien el cuadrado de lado $1$ en el plano $x-y$ pero con la orientacion opuesta porque si primero camino en direccionde $e_2$ y luego de $e_1$ el vector de velocidad angular va en direccion de $z$-negativo.
            
            \item $\bigwedge^0 V = \RR$ son puntos (0-paralelepipedos) con tamaño.
        
            \end{itemize}  
        
        \textbf{El producto cuña} es entonces una \textbf{generalizacion del producto cruz: el producto cruz} usual coge $2$ $1$-paralelepipedos/vectores $v, w$ y les asigna el $2$-paralelepipedo orientado / area orientada / plano orientado  $v \wedge w$ identificado por su vector perpendicular $v \times w$ cuya magnitud es igual al area ($|v||w| sin(\theta)$) con signo del paralelepipedo $v \wedge w$! Otro ejemplo ilustrativo es el producto cuña entre un vector y un area orientada es un volumen orientado, y vemos que si el vector es paralelo a esta area orientada, entonces el volumen orientado obtenido es $0$, asi como el producto cuña entre vectores paralelos es $0$.
        
        \item Entendiendo asi el producto cuña, entender a $\clvg$ como el subalgebra de $End(\extv)$ tal que \dots es lo mismo que decir que puedo multiplicar a $a \in \clvg$ por un elemento de $\extv$. En particular, \textbf{multiplicar $v \in V \subset \clvg$ al vector $w \in V$, es decir ``multiplicar'' $v$ con $w$ nos da \[vw = \epsilon(v)w + \iota(v)w = v \wedge w + v \cdot w "= v \times w + v \cdot w "\]} De hecho, esto es mas general: multiplicar a $v \in V \subset \clvg$ por $v_1 \wedge \cdots \wedge v_k$ es una version general de ``p. cu\~na/cruz + p. punto'' $v \,v_1 \wedge \cdots \wedge v_k = v \wedge v_1 \wedge \cdots \wedge v_k + meet(v_1, v_1 \wedge \cdots \wedge v_k)$\\ = $\epsilon(v) v_1 \wedge \cdots \wedge v_k + \iota(v) v_1 \wedge \cdots \wedge v_k$
        
        \item La parte par de $Cl(\RR^2)$ es isomorfa a $\CC$.
        
        \item La parte par de $Cl(\RR^3)$ es isomorfa a $\HH$ (y por muy buenas razones, casi que de aqui surgieron las algebras de Clifford: como algebras geometricas; H nacio para ser El Grupo de Rotaciones de $\RR^3$ asi como $\CC$ es el grupo de rotaciones de $\RR^2$)
        
        \item $Cl(\RR^3)$ es isomorfo como algebra a $M_2(\CC)$ donde el isomorfismo está inducido por el mapa $\RR^3 \to M_2(\CC)$, $e_1 \mapsto \sigma_i$... esto lo sabiamos/intuíamos del hecho de que las matrices de Pauli satisfacieran $\set{\sigma_i, \sigma_j} = 2 \delta_{ij}$
        
        \item $Cl(\RR^{1, 3}) = ?$... no se, solo se que su parte par es isomorfa a $Cl(\RR^3)$.
            
        \end{itemize}
\end{frame}



%%%%%%%%%%%%%%%%%%%%%%%%%%%%%%%%%%%%%%%%%%%%%%%%%%%%%%%%%%%%%%%%%%%%%%%%%%%%%%%
%%%%%%%%%%%%%%%%%%%%%%%%%%%%%%%%%%%%%%%%%%%%%%%%%%%%%%%%%%%%%%%%%%%%%%%%%%%%%%%
\section{Spinors}

%%%%%%%%%%%%%%%%%%%%%%%%%%%%%%%%%%%%%%%%%%%%%%%%%%%%%%%%%%%%%%%%%%%%%%%%%%%%%%%
\subsection{Spin Groups}

\begin{frame}{From now on\dots} % % % % % % % % % % % % % % % % % % %

    Let $n = dim_\RR V$, and $m = \lfloor \frac{n}{2} \rceil$.

    From now on we assume that \emph{$g$ is positive definite}, so that $V \cong \RR^n$ as inner-product vector spaces, and so we may write $v_1 \cdot v_2 := g(v_1, v_2)$ for $v_i \in V$. %(even in the big book Elements of Noncommutative Geometry this is assumed). 
    
    Let $\set{e_1, \dots, e_n}$ be an \emph{oriented} orthonormal basis.
    
    $\gamma = (-1)^m e_1 \cdots e_n$, but it is independent of the chosen oriented orthonormal basis. % In \RR^n: that the determinant of the matrix of columns is 1, not -1.

    We will concentrate on the \textbf{complex} Clifford algebra $\CC l(V)$, with hermitian inner product $\bra \cdot , \cdot \ket$, so we will simply call it simply \emph{the Clifford algebra}.% Even if g is not positive definite, it is ISOMORPHIC, but not equal to the positive-definite case.
    
\end{frame}

\begin{frame}{Unit Vectors as Perpendicular Reflections} % % % % % % % % % % % % % % % % % % %
    Let \defn{$\cclv^\times$} be the group of invertible elements.
        \begin{itemize}
            
        \item Notice that, if $v \in V$, by the defining property of the Clifford algebra, $v^2 = ||v||^2$, we have that $v^\inv = \frac{v}{||v||^2}$.
            
        \end{itemize}
    
    Call an element $u \in \cclv$ \defn{unitary} if $u^* u = 1$. They are such that $\bra a, b \ket = \tau(a^* b) = \tau(a^*(u^* u) b)  = \bra ua, ub \ket$.
        \begin{itemize}
            
        \item If $x \in V$ is of unit length,  $x^* = \conj{(x^t)} = x = x^
        \inv$.
        
        \item Unitary vectors $w$ in $V^\CC$ \Iff $w = \lambda x$, where $\lambda \in U(1) =: \defn{\TT}$, $x$ is a unit vector in $V$.
        
        \item If $u$ is unitary \then $u \in \cclv^\times$.
            
        \end{itemize}
    

\end{frame}

\begin{frame} % % % % % % % % % % % % % % % % % % %

    For $a \in \cclv^\times$, let \defn{$Ad_a$} be the map $b \mapsto a b a^\inv$ for $b \in \cclv$. 
        \begin{itemize}
            
        \item This is a group representation, i.e. $Ad_{aa'} = Ad_a \circ Ad_{a'} \in GL(\cclv)$ 
            
        \end{itemize}

    \begin{lemma}
    For $x \in $ of unit length and $v \in V$, $Ad_x(v) = v - 2 (x \cdot v) = $ ``\textbf{Reflection} of $v$ with respect to to the plane orthogonal to $x$''. So, $Ad_x \in O(V, g)$ when applied to $V\subset \cclv$.
    \end{lemma}
    \begin{proof} Hola
    \end{proof}
    
    Finally, it is not hard to see that \textbf{applying $2$ / an even amount of reflections is a rotation: $Ad_{x_1 x_2}\in SO(V, g)$}; \\the converse is also true: every rotation of $V$ may be written as an even product of reflections.
    
\end{frame}

\begin{frame}{$Spin(V)$} % % % % % % % % % % % % % % % % % % %

    \begin{Definition} $\defn{\spinv} := \set{x_1 \cdots x_{2k}: x_i \in V, ||x_i||^2 = x_i^* x_i = 1} \subset \clvg \subset \cclv$
    \end{Definition}
    
    We saw that $\spinv$ acts on $V$, by the adjoint representation, by rotations. Furthermore:
    
    \begin{proposition}
    The map $Ad|_{\spinv}: \spinv \to SO(V, g)$ is surjective.
    
    For any element $u \in \spinv$, both and $u, -u$ map to the same rotation.
    \end{proposition}
    
    TODO: it may be charactezed by $u \in \cclv$ is in $\spinv$ \iff $u$ is even, unitary and real ($u = \conj u$).
\end{frame}

\begin{frame}{Example: $\spin(2)$} % % % % % % % % % % % % % % % % % % %
    
    To see: \[\spin(2) = \set{x + y e_1 e_2 \,|\, x, y \in \RR;  x^2 + y^2 = 1} \cong \TT\]
    
    $\spin(2) \subset \set{x + y e_1 e_2 \,|\, x, y \in \RR;  x^2 + y^2 = 1}$:
        \begin{itemize}
            
        \item Product of $2$ unit vectors $\subset \set{x + y e_1 e_2}$: Let $u_1 = a e_1 + b e_2$ and $u_2 = \alpha e_1 + \beta e_2$ be normalized vectors in $\RR^2$. Then $u_1 u_2 = x + e_1 e_2 y \in \spin(2)$, with $x = a \alpha + b \beta$ and $y = a \beta  - b \alpha$ and $x^2 + y^2 = (a^2 + b^2)(\alpha^2 + \beta^2) = 1$.
        
        \item Even product of unit vectors $\subset$ Product of elements of $\set {x + y e_1 e_2} \subset \set {x + y e_1 e_2}$: $(x_1 + y_1 e_1 e_2)(x_2 + y_1 e_1 e_2) = x + y e_1 e_2$ with $x = x_1 y_1 - x_2 y_2$ and $y = x_1 y_2 + x_2 y_1$ and so $x^2 + y^2 = 1$.
            
        \end{itemize}
    
\end{frame}

\begin{frame}{Example (Cont.): $\spin(2)$} % % % % % % % % % % % % % % % % % % %
    
    (Cont.) To see: \[\spin(2) = \set{x + y e_1 e_2 \,|\, x, y \in \RR;  x^2 + y^2 = 1} \cong \TT\]
    
    $\spin(2) \subset \set{x + y e_1 e_2 \,|\, x, y \in \RR;  x^2 + y^2 = 1}$:
        
        \begin{itemize}
        
        \item Any element $x + y e_1 e_2$ can be seen as the product of unit vectors $e_1$ and $x e_1 + y e_2$: $e_1 (x e_1 + y e_2) = x e_1^2 + y e_1 e_2 = x + y e_1 e_2 \in \spin(2)$
            
        \end{itemize}
    \qed
    
    Looking at $\spin(2)$ as a ``subset'' of $M_2(\RR) \cong Cl(2)$, then $\spin(2) \cong \set{\begin{pmatrix} x & -y \\ y & x \end{pmatrix} \,|\, x^2 + y^2 = 1} \subset M_2(\RR)$.
    
\end{frame}

\begin{frame}{Example: $SU(3)$} % % % % % % % % % % % % % % % % % % %

    The Lie algebra of $Spin(3) \subset Cl(\RR^3)$ is $Q(\bigwedge^\bullet \RR^3) = Cl^0(\RR^3) = \HH$, where $\hat i = e_2 e_3, \hat j = e_3 e_1, \hat k = e_1 e_2$.
    
    $Cl(\RR^3) \cong M_2(\CC)$ via the identification $e_j \mapsto \sigma_j$; so $\hat i \mapsto i \sigma_1, \hat j \mapsto i \sigma_2, \hat i \mapsto k \sigma_3$.
    
    The Lie algebra of $SU(2)$ is precisely that generated by $i\sigma_1, i\sigma_2, i\sigma_3$: they satisfy the commutation relations $[i\sigma_i, i \sigma_j] = 2 \epsilon_{ijk} (i \sigma_k)$; every element of $SU(2)$ may be written as $exp\{\frac{1}{2} \phi \, \vec n \cdot \vec{i \sigma}\}$.
    
    So, under the isomorphism $\CC l(\RR^3) \cong M_2(\CC) \oplus M_2(\CC)$ the Lie algebra of $\spin(3)$ is identified with the Lie algebra of \rtext{$SU(2)$ \Then $Spin(3) \cong SU(2)$}.
    
\end{frame}

\begin{frame}{$Spin^c(V)$} % % % % % % % % % % % % % % % % % % %

    \begin{Definition} $\defn{\spincv} := \set{u_1 \cdots u_{2k}: u_i \in V^\CC, u^*u = 1} \subset \cclv$
    \end{Definition}
    
    
    \begin{proposition}
    The map $Ad|_{\spincv}: \spincv \to SO(V, g)$ is surjective.
    
    For any element $u \in \spincv$, both all of $\TT u$ map to the same rotation.
    \end{proposition}
    
    $Spin(V)$ is clearly a subgroup. Furthermore,  defining $\nu: \spinc \to \TT$, $u \mapsto u^! u$, we may identify $ker\,\nu = \spinv$, because that implies that $u^*= u^!$, so $u = \conj u$.
    
    $\spincv \cong \spinv \times \TT / \{\pm 1\}$, via the surjective group homomorphism $(Ad|_{\spincv}, \nu): \spincv \to \spinv \times \TT$.
    
\end{frame}

%%%%%%%%%%%%%%%%%%%%%%%%%%%%%%%%%%%%%%%%%%%%%%%%%%%%%%%%%%%%%%%%%%%%%%%%%%%%%%%
\subsection{Spinor Space / Fermionic Fock Space}

\begin{frame}{Complex Structure $J$ and $V$ as a complex vector space} % % % % % % % % % % % % % % % % % % %
    
    Suppose that $V$ be even dimensional%, so $V \cong \CC^m$, but not in a canonical way
    . A complex structure $J$ is what allows the identification of $V \cong_J \CC^m$ as complex vector spaces.
    
    \begin{definition}
    An operator $\defn{J} \in End_\RR(V)$ is an \emph{orthogonal complex structure on V} if 
        
        \begin{itemize}
            
        \item $J^2 = -1$
        
        \item $g(Ju, Jv) = g(u, v)$ for all $u, v \in V$
            
        \end{itemize}
    \end{definition}
    
    This allows to think of $V$ as an $m$ dimensional complex vector space where $i \cdot v := Jv \in V$.
    
    \emph{Remark}: This multiplication of $v$ by $i$ is different to the multiplication by $i$ in $V^\CC = V \oplus iV$, because the latter takes us outside $V$ into $iV$.

\end{frame}

\begin{frame}{Example: $\RR^{2m}$} % % % % % % % % % % % % % % % % % % %
    
    In $\RR^{2m}$, usually 
    $J := \begin{pmatrix} 
        I_n & 0_n \\
        0_n & -I_n
    \end{pmatrix}$.
    
    In $\RR^2$ this means that $i \twov{x}{y} := \twov{-y}{x}$, so we are making the natural identification of $\RR^2 \ni \twov{x}{y} \mapsto z = x + iy \in \CC^1$ as complex vector spaces.
    
    Having chosen a complex structure $J$, we may define the \emph{sesquilinear} inner product \[ \defn{\bra} u, v \defn{\ket_J} := g(u, v) + ig(Ju, v) \]
    
    With this complex inner product, we will denote by \defn{$V_J$} the $m$-dimensional \textbf{(complex) Hilbert space} $(V, \bra \cdot, \cdot \ket_J)$.
    
    
    %(Do not talk about $W_J$ here, but in the section of Polarized Fock Space. Con: can't talk of the action of $w \in W_J$, which is used to prove that the action is self adjoint; can't proof important things)... Creo que mejor si, para poder probar al menos que la rep. en el caso par es irrep, y autoadjunta: $W_J$, $P_J$ 
    
\end{frame}

\begin{frame}{$W_J$} % % % % % % % % % % % % % % % % % % %
    For our purposes, this space will be useful to prove things about representations easily.
    
    $J$ is an operator $:V \to V$ in contrast to the ``operator'' $i: V\subset V^\CC \to iV \subset V^\CC$ and $i:iV \to V$.
    
    \begin{definition}
    Given an orthogonal complex structure $J$ on $V$, define $\defn{P_J}: V_J \to V^\CC$ by $v \mapsto \frac{1}{2}(1 - iJ)v = \frac{1}{2}(v - \frac{1}{2}iJv)$. We will usually denote by $\defn{w_z} := P_J(v) \in W_J$.\\
    Define $\defn{W_J} \subset V^\CC$ as the image of $P_J$. We can see that $W_J = \set{v - iJv \,|\, v \in V}$. It is a \textbf{complex vector subspace of $V^\CC$}.\\
    Define $\defn{\conj{W_J}} \subset V^\CC$ as the image of $P_{-J}$. We see that $\conj{W_J} = \set{v + iJv \,|\, v \in V}$.
    \end{definition}
    
    Notice that $V^\CC = W_J \oplus \conj{W_J}$, e.g. $V \ni v = \frac{1}{2}(w_v  + \conj{w_v})$.
    
\end{frame}

\begin{frame}{Hilbert space structure of $V^\CC$} % % % % % % % % % % % % % % % % % % %
    
    In general, we can make $V^\CC$ a (complex) Hilbert space by ``twisting'' $g^\CC$ \[ \defn{\bbra} \alpha, \beta \defn{\kket} := 2 g^\CC(\conj{\alpha}, \beta) \qquad \textbf{for } \alpha, \beta \in V^\CC \]
    
    $W_J$ and $\conj{W_J}$ are orthogonal with respect to this product, so we say that $W_J$ is a \defn{polarization of $V^\CC$}.
    
    When we restrict $\bbra \cdot , \kket$ to $W_J$ \[P_J: (V_J, \bra \cdot, \cdot \ket_J) \to (W_J, \bbra \cdot, \cdot \kket) \] \textbf{is a unitary ($\bra u, v \ket_J = \bbra w_u, w_v \kket$) isomorphism}.
    
\end{frame}

\begin{frame}{Example: $\RR^4$} % % % % % % % % % % % % % % % % % % %

In $\RR^4$, recall that, e.g. $J(e_1) = e_3$ and $J(e_3) = -e_1$.
\[
    W_J = span_\CC\set{e_1 -i e_3, e_2 - ie_4} \subset{\RR^4 \otimes \CC}
\]

Notice that $i(e_1 - i e_3) = e_3 + ie_1 = e_3 - i (-e_1) = e_3 - i (J e_3)$, so $W_J$ is indeed a complex vector subspace unlike $V$.

\end{frame}

%%%%%%%%%%%%%%%%%%%%%%%%%%%%%%%%%%%%%%%
\subsubsection{Representation of $\CC l(V)$: $V$ even-dimensional}

\begin{frame}{Fock and Polarized Fock Space: even dimension} % % % % % % % % % % % % % % % % % % %

    \begin{definition}
    \end{definition}
    
    Inner product

\end{frame}

\begin{frame}{Theorems about Fock Space: even dimension} % % % % % % % % % % % % % % % % % % %

    Theorem: $c_J$ is a selfadjoint representation of the *-algebra $\CC l(V)$: comes ``easily'' from knowing what the actio of $w \in W_J$ is.
    
    Theorem (Varilly chiquito): It is the \textbf{unique} \emph{irreducible} algebra representation of $\CC l(V)$: irrep comes immediately from Schur's lemma
    
    Lemma: $\gamma$ acts as the grading operator on the Fock space, which is the exterior algebra grading. ... useful, perhaps, because of teoremita de orden normal? -> \textbf{2 types of spinors} (which act differently by Spin, so act differently by SL( . , C)?)

\end{frame}

\begin{frame}{Example of Fock Space} % % % % % % % % % % % % % % % % % % %

    Example: Sanchez. Chirality is what we expect.

\end{frame}

%%%%%%%%%%%%%%%%%%%%%%%%%%%%%%%%%%%%%%%
\subsubsection{Representation of $\CC l(V)$: $V$ odd-dimensional}

\begin{frame}{Fock Space: odd dimension} % % % % % % % % % % % % % % % % % % %

    Paragraph of Varilly grande.
    
    Many possibilities for the FOck space, as there are many ``orthonormal hyperplanes''
    
    In general: $dim_\CC (\mathcal F(V)) = 2^{\lfloor\frac{dim_\RR V}{2} \rfloor}$
    
    Theorem: 
        \begin{itemize}
        
        \item There are exactly 2 unitarily inequivalent irreps, both can be seen on Fock space. (Varilly chiquito)
        
        \item The 2 irreps agree in the even part of the algebra.
        
        \end{itemize}

\end{frame}

%%%%%%%%%%%%%%%%%%%%%%%%%%%%%%%%%%%%%%%
\subsubsection{Spin Groups Representations}

\begin{frame}{Results} % % % % % % % % % % % % % % % % % % %

    Even:
        \begin{itemize}
            
        \item $Spin^C$ and $Spin$ act irreducibly and inequivalently on the positive (chirality) and negative (chirality) of Fock space. \textbf{i.e. there are 2 types of spinors, each of which acts differently under Spin, so under Lorentz $\subset SL(, \CC)$}!!
            
        \end{itemize}
    
    Odd:
        \begin{itemize}
        
        \item Although there are $2$ inequivalent irreps of the algebra $\CC l(V)$ which can both be seen in the same Fock space, $Spin^\CC$ and $Spin$ act equivalently and irreducibly in this Fock space. TODO: que significa?
        
        \end{itemize}

\end{frame}


%%%%%%%%%%%%%%%%%%%%%%%%%%%%%%%%%%%%%%%%%%%%%%%%%%%%%%%%%%%%%%%%%%%%%%%%%%%%%%%
%%%%%%%%%%%%%%%%%%%%%%%%%%%%%%%%%%%%%%%%%%%%%%%%%%%%%%%%%%%%%%%%%%%%%%%%%%%%%%%
\section{Further Topics}

%%%%%%%%%%%%%%%%%%%%%%%%%%%%%%%%%%%%%%%%%%%%%%%%%%%%%%%%%%%%%%%%%%%%%%%%%%%%%%%
\subsection{Quantum Statistical Mechanics?}

%%%%%%%%%%%%%%%%%%%%%%%%%%%%%%%%%%%%%%%%%%%%%%%%%%%%%%%%%%%%%%%%%%%%%%%%%%%%%%%
\subsection{Spin Bundles and Dirac Operators}

\begin{frame}{Frame Title} % % % % % % % % % % % % % % % % % % %

\end{frame}



\end{document}
